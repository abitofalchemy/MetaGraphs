%% Note to self
%% http://www.sciencemag.org/site/feature/data/compsci/machine_learning.xhtml

\part{Outline}
\section{Introduction}

\subsection{Context}
\subsection{The Need or Research Questions}
\subsubsection{What we have}
Building on the shoulders of
\subsubsection{What we need/want }
\subsection{Task}
What we have done to date.
\subsection{Object of the document}
Our contributions
\part{Summary}


\textbf{Context. }
Wikipedia and similar user generated knowledge-bases evolve a function of new
users joining and contributing over time.  Page articles are created and edited
until the article is complete, conflict or controversies are resolved.

The dynamic of evolving networks sheds light into how humans acquire,
document, and consume knowledge. Our aim is to systematically learn how network
formation takes place.  We want to learn the rules that help predict network
evolution or network formation.  We want to learn graph grammars for networks
in action (dynamic networks).

We start out with static snapshots of network graphs by inferring a set of
rules that predicts the graph as a function of node count (or as function of time).

\textbf{Task. } We have 

\textbf{Object of the article. } This paper describes a new approach to learn
graph grammars or production rules given large network graphs.
